\documentclass[%twocolumn,
 reprint,onecolumn,
%superscriptaddress,
%groupedaddress,
%unsortedaddress,
%runinaddress,
%frontmatterverbose, 
%preprint,
%showpacs,preprintnumbers,
%nofootinbib,
%nobibnotes,
%bibnotes,
 amsmath,amssymb,
% prl,
aps,
%pra,
%prb,
%rmp,
%prstab,
%prstper,
%floatfix,
]{revtex4-1}


\usepackage{cancel}
\usepackage{enumitem}
\usepackage[utf8x]{inputenc}
\usepackage{graphicx}% Include figure files
\usepackage{dcolumn}% Align table columns on decimal point
\usepackage{bm}% bold math
\usepackage{hyperref}% add hypertext capabilities
\usepackage{tikz}
\usepackage{subfigure}
\setlength\parindent{0pt}
%\usepackage[switch]{lineno}% Enable numbering of text and display math
%\linenumbers % Commence numbering lines

%\usepackage[%showframe,%Uncomment any one of the following lines to test 
%%scale=0.7, marginratio={1:1, 2:3}, ignoreall,% default settings
%%text={7in,10in},centering,
%%margin=1.5in,
%%total={6.5in,8.75in}, top=1.2in, left=0.9in, includefoot,
%%height=10in,a5paper,hmargin={3cm,0.8in},
%]{geometry}

\begin{document}

\section{1D Harmonic Oscillator}

Let us consider a mono-dimensional system composed by a single particle with mass $m$. The particle position is described by the cartesian coordinate $x$ and its momentum by $p$. 
Assuming that the particle experiences an harmonic potential with equilibrium position at $x=0$, the system's Hamiltonian can be written as
%
\begin{equation}
H_{1D}=\frac{p^{2}}{2m}+\frac{1}{2}m\omega x^{2}\:,
\label{Ham}
\end{equation}
%
where $\omega$ is the typical harmonic vibrational frequency. 

This Hamiltonian is often transformed in an equivalent form by mean of a coordinates transformation that renders $x$ and $p$ unit-less. 
The new position and momentum $\bar{x}$ and $\bar{p}$ are defined as
%
\begin{equation}
\bar{p}=\sqrt{\frac{1}{m\hbar\omega}}p \:,\quad \bar{x}=\sqrt{\frac{m\omega}{\hbar}}x  \:.
\end{equation}
%

The Hamiltonian \ref{Ham} in this new basis becomes 
%
\begin{equation}
H_{1D}=\frac{\hbar\omega}{2}\large( \bar{p}^{2} + \bar{x}^{2} \large) \:,
\label{Ham2}
\end{equation}
%
It is also convenint to introduce creation and annihilation operators 
%
\begin{equation}
a^{\dag}=\frac{1}{\sqrt{2}}\large(\bar{x}-i\bar{p}\large) \:\quad \text{and}\quad  a=\frac{1}{\sqrt{2}}\large(\bar{x}+i\bar{p}\large)\:,
\label{ladder}
\end{equation}
%
with commuting rule $[a,a^{\dag}]=1$ and by means of which we finally obtain a third, equivalent, form of the system Hamiltonian
%
\begin{equation}
H_{1D}=\hbar\omega(\hat{n}+\frac{1}{2}) \:,
\label{Ham3}
\end{equation}
where $\hat{n}=a^{\dag}a$ is the particle number operator. 



\section{DFT - N-1D Harmonic Oscillators Mapping}


A molecular system made by $N$ interacting particles can be described in the harmonic approximation assuming the potential energy surface $U$ to be well described by its Taylor expansion around the $T=0 K$ equilibrium position. In this case 
%
\begin{equation}
U(\vec{\mathbf{x}})=\frac{1}{2}\sum_{ij}^{3N}\frac{\partial E_{el}}{\partial x_{i}\partial x_{j}}x_{i}x_{j}\:,
\end{equation} 
%
where $E_{el}$ is the adiabatic electronic energy and $\vec{\mathbf{x}}$ is a $3N$ dimensional vector describing the system configuration. 
This system can still be mapped on a set of 3N decoupled 1D harmonic oscillators by introducing the normal mode of vibration. We first start defining mass-weighted cartesian coordinates $u_{i}=\sqrt{m_{i}}x_{i}$ and by diagonalizing the force-constant matrix of the energy second-order derivatives $\mathbf{H}$
%
\begin{equation}
\frac{1}{2}\sum_{ij}^{3N}\frac{\partial E_{el}}{\partial x_{i}\partial x_{j}}x_{i}x_{j}=\frac{1}{2}\sum_{ij}^{3N}\frac{\partial E_{el}}{\partial u_{i}\partial u_{j}}u_{i}u_{j}=\frac{1}{2}\sum_{ij}^{3N}H_{ij}u_{i}u_{j} \rightarrow \frac{1}{2}\sum_{i}^{3N}\textit{diag}(\mathbf{H})_{ii}q^{2}_{i}
\label{diagH}
\end{equation}
%
The system Hamiltonian can now be written as is eq. \ref{Ham} by mapping the $q_{i}$ coordinates, expressed by the $\mathbf{H}$ eigenvectors $L_{ij}$, on the 1D mass-weighted cartesian coordinates $\sqrt{m}x$ of last section. Doing so, the frequencies of normal vibrations are defined as $\omega_{i}=\sqrt{\textit{diag}(\mathbf{H})_{ii}}$ while $q_{i}$ coordinates are defined as function of cartesian coordinates $x_{j}$ through $\mathbf{H}$ eigenvectors $L_{ij}$
%
\begin{equation}
q_{j}=\sum_{i}^{3N}L_{ij}\sqrt{m_{i}}x_{i}
\end{equation}
%
As done in the previous section we here define a set of unit-less normal modes $\bar{q_{i}}$
\begin{equation}
\bar{q}_{j}=\sqrt{\frac{\omega_{j}}{\hbar}}\sum_{i}^{3N}L_{ij}\sqrt{m_{i}}x_{i}
\end{equation}
and the inverse transformation that defines the cartesian displacement associated to a unit-less normal mode $\bar{q}_{j}$ amount of displacement
%
\begin{equation}
\bar{x}_{i}=\sqrt{\frac{\hbar}{m_{i}\omega_{j}}}L_{ij}\bar{q}_{j}\:.
\end{equation}
%
Finally, we can also write the i- unit-less normal mode as function of creation and annihilation operators as
%
\begin{equation}
\bar{q}_{i}=\frac{1}{\sqrt{2}}\large(a^{\dag}_{i}+a_{i}\large).
\end{equation}
%
with the usual commutation rule $[a_{i},a_{j}^{\dag}]=\delta_{ij}$ and total Hamiltonian
%
\begin{equation}
H_{N}=\sum_{i}\hbar\omega(\hat{n}_{i}+\frac{1}{2}) \:,
\end{equation}
where $\hat{n}_{i}=a_{i}^{\dag}a_{i}$ is the number operator for the i-normal mode. 


\section{Redfield Equations}

\subsection*{First order in coupling strength and first order in time-dependent perturbation theory}

\subsection*{Second order in coupling strength and first order in time-dependent perturbation theory}

\subsection*{First order in coupling strength and second order in time-dependent perturbation theory}








\end{document}
